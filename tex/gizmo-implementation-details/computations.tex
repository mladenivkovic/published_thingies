%=========================================
\section{Computations}
%=========================================





%=========================================
\subsection{Normalization}
%=========================================


To compute the normalisations \ref{omega}, we need to sum over all neighbouring particles and sum the kernels correctly.
To evaluate the kernels, we use the \verb|kernel_deval(xij, wij, wij_dx)| function in SWIFT.

If a kernel is defined as
\begin{align*}
	W_i(\x) = W(\x - \x_i, h(\x)) = \frac{1}{h(\x)^\nu} w\left(\frac{| \x - \x_i |}{h(\x)} \right)
\end{align*}

then \verb|kernel_deval| computes 

\begin{align*}
	\mathtt{wij} &= w(\mathtt{xij}) \\
	\text{and \quad} \mathtt{wij\_dx} &= \frac{\del w(r)}{\del r} \big{|} _{r = \mathtt{xij}} \\ 
	\text{with \quad} \mathtt{xij} &= \frac{| \x_i - \x_j |}{h(\x_i)}
\end{align*}



So for a specific particle $i$, we need to compute
\begin{align*}
	\omega(\x_i) 	&= \sum_j W(\x_i - \x_j, h(\x_i)) \\
					&= \sum_j \frac{1}{h(\x_i)^\nu} w\left( \frac{| \x_i - \x_j |}{h(\x_i)} \right) \\
					&= \sum_j \frac{1}{h_i^\nu} \ \mathtt{ wij }
\end{align*}


with $h_i = h(\x_i)$.








%================================================
\subsection{Analytical gradients of $\psi(\x)$}
%================================================

For the \cite{ivanovaCommonEnvelopeEvolution2013} expression of the effective surfaces \Aij, we need analytical gradients in Cartesian coordinates of $\psi_i(\x_j)$.

From eq. \ref{psi} we have that
\begin{align*}
	\psi_j(\x_i) &= \frac{ W(\x_i - \x_j, h(\x_i))}{\omega(\x_i)}\\
\end{align*}

Let $r_{ij} \equiv |\x_i - \x_j|$ and $q_{ij} \equiv \frac{r_{ij}}{h_i}$. Then

\begin{align}
	\deldx \psi_j(\x_i) 	&= \deldx \frac{ W(\x_i - \x_j, h(\x_i))}{\omega(\x_i)} 
							=  \deldx \frac{ W(r_{ij}, h_i)}{\omega(\x_i)}		\nonumber \\
%
							&= \frac{ 
									\DELDX{W}(r_{ij}, h_i) \  \omega(\x_i) -
									W(r_{ij}, h_i) \DELDX{\omega}(\x_i)
									}{\omega(\x_i)^2}		\nonumber  \\
%
							&= \frac{1}{\omega(\x_i)} \DELDX{W}(r_{ij}, h_i) -
							\frac{1}{\omega(\x_i)^2} W(r_{ij}, h_i) \DELDX{}\sum_k W(r_{ik}, h_i)		\nonumber \\
%
							&= \frac{1}{\omega(\x_i)} \DELDX{W}(r_{ij}, h_i) -
									\frac{1}{\omega(\x_i)^2} W(r_{ij}, h_i) \sum_k \DELDX{W}(r_{ik}, h_i)	\label{grad_psi}
\end{align}






If a kernel\footnote{
Helpful way to think of the indices: $W_j(\x_i) = W(|\x_j - \x_i|, h_i)$ is the kernel value of particle $i$ at position $\x_i$ evaluated at the position $\x_j$.
Personally I would've used the indices the other way around for simplified thinking, but I'm going to stick to this notation because Hopkins also uses it.
}
is defined as

\begin{align*}
	W_j(\x_i)	 = W(\x_i - \x_j, h_i))
				 = \frac{1}{h_i^\nu} w \left(\frac{| \x_i - \x_j |}{h_i} \right)
				 =  \frac{1}{h_i^\nu} w ( q_{ij} )
\end{align*}





and we assume that the smoothing length $h_i$ is treated as constant at this point, then the gradient of the kernel is given by

\begin{align}
	\deldx W_j (\x_i) 	&= \deldx \left( \frac{1}{h_i^\nu} w ( q_{ij} )	\right)	\nonumber \\
						&= \frac{1}{h_i^\nu} \frac{\del w(q_{ij})}{\del q_{ij}} \frac{\del q_{ij}(r_{ij})}{\del r_{ij}} \frac{\del r_{ij}}{\del x} 	\label{grad_kernel}
\end{align}




We now use
\begin{align}
	\frac{\del q_{ij}(r_{ij})}{\del r_{ij}} 	&= \frac{\del}{\del r_{ij}} \frac{r_{ij}}{h_i} = \frac{1}{h_i}		\label{dqdr} \\
	\DELDX{r_{ij}}		&= \deldx \sqrt{(\x_i - \x_j)^2}
							= \frac{1}{2} \frac{1}{\sqrt{(\x_i - \x_j)^2}} \cdot 2 (\x_i - \x_j) \nonumber \\
						&= \frac{\x_i - \x_j}{r_{ij}} 	\label{drdx}
\end{align}











To be perfectly clear, we should in fact write
\begin{align*}
	r_j(\x) & \equiv | \x - \x_j | \\
\end{align*}

which again leads to

\begin{align*}
	\DELDX{r_{ij}}		&= \DELDX{r_j(\x_i)} = \DELDX{r_j(\x)} \bigg{|}_{\x = \x_i} \\
						&= \deldx \sqrt{(\x - \x_j)^2} \big{|}_{\x = \x_i}
							= \frac{1}{2} \frac{1}{\sqrt{(\x - \x_j)^2}} \cdot 2 (\x - \x_j)  \big{|}_{\x = \x_i} \\
						&= \frac{\x - \x_j}{r_{j}(\x)} \big{|}_{\x = \x_i} 
							= \frac{\x_i - \x_j}{r_{ij}}
\end{align*}



















Inserting expressions \ref{dqdr} and \ref{drdx} in \ref{grad_kernel}, we obtain
\begin{equation}
	\deldx W_j (\x_i)  = \frac{1}{h_i^{\nu + 1}} \frac{\del w(q_{ij})}{\del q_{ij}}  \frac{\x_i - \x_j}{r_{ij}} 	\label{grad_kernel_final}
\end{equation}



$\frac{\del w(q_{ij})}{\del q_{ij}}$ is given by \verb|wij_dx| of \verb|kernel_deval|.







Finally, inserting \ref{grad_kernel_final} in \ref{grad_psi} we get
\begin{equation}
	\deldx \psi_j (\x_i) = 
		\frac{1}{\omega(\x_i)} \frac{1}{h_i^{\nu + 1}} \mathtt{wij\_dx}  \frac{\x_i - \x_j}{r_{ij}} 	 -
		\frac{1}{\omega(\x_i)^2} W(r_{ij}, h_i) \sum_k \frac{1}{h_i^{\nu + 1}} \mathtt{wik\_dx}  \frac{\x_i - \x_k}{r_{ik}}
\end{equation}






















The definition of $r_{ij}$ requires a bit more discussion.
Since kernels used in hydrodynamics (at least in those methods currently implemented in SWIFT) are usually taken to be spherically symmetric, we might as well have defined

\begin{align*}
r'_{ij} = |\x_j - \x_i |
\end{align*}

which would leave the evaluation of the kernels invariant [$r'_{ij} = r_{ij}$], but the gradients would have the opposite direction:

\begin{align*}
\DELDX{r'_{ij}}	
=   \frac{\x_j - \x_i}{r'_{ij}} 
= - \frac{\x_i - \x_j}{r_{ij}}
= - \DELDX{r_{ij}}
\end{align*}

So which definition should we take? \\


Consider a one-dimensional case where we choose two particles $i$ and $j$ such that $x_j > x_i$ and $q_{ij} = | x_j - x_i | / h_i < H$, where $H$ is the compact support radius of the kernel of choice.
Because we're considering a one-dimensional case with $x_j > x_i$, we can now perform a simple translation such that particle $i$ is at the origin, i.e. $x'_i = 0$ and $x'_j = x_j - x_i = | x_j - x_i | = r_{ij}$.
In this scenario, the gradient in Cartesian coordinates and in spherical coordinates must be the same:


\begin{align}
	\frac{\del}{\del x'} W (|x_i' - x'|, h_i)  \big{|}_{x' = x_j' } &= \frac{\del}{\del r_{ij}} W (r_{ij}, h_i) 		\nonumber\\
%
	\Rightarrow \quad \frac{1}{h_i^\nu} \frac{\del w(q'_{ij})}{\del q'_{ij}} \frac{\del q'_{ij}(r'_{ij})}{\del r'_{ij}} \frac{\del r'_{i}(x')}{\del x'}	  \big{|}_{x' = x_j' }  &=
		\frac{1}{h_i^\nu} \frac{\del w(q_{ij})}{\del q_{ij}} \frac{\del q_{ij}(r_{ij})}{\del r_{ij}}				\label{spher_cart}
\end{align}




We have the trivial case where
\begin{align*}
	r'_{ij} &= |x_i' - x_j'| = | x_i - x_j | = r_{ij} \\
	q'_{ij} &= r'_{ij}/h_i = q_{ij} \\
	\Rightarrow \quad \frac{\del w(q'_{ij})}{\del q'_{ij}} &= \frac{\del w(q_{ij})}{\del q_{ij}}, \quad \frac{\del q'_{ij}(r'_{ij})}{\del r'_{ij}} = \frac{\del q_{ij}(r_{ij})}{\del r_{ij}}
\end{align*}



giving us the condition from \ref{spher_cart}:






\begin{equation}
	\frac{\del r'_{i}(x')}{\del x'}	  \big{|}_{x' = x_j' } = 1 = \frac{\del r_{i}(x)}{\del x}
\end{equation}


this is satisfied for
\begin{align*}
	r_j(\x) &= | \x - \x_j | \\ 
	\Rightarrow r_{ij} &= | \x_i - \x_j |,\ \text{ not } \ r_{ij} = | \x_j - \x_i |
\end{align*}
















